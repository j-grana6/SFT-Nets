\documentclass{article}
\usepackage{amsmath}
\usepackage[margin=1in]{geometry}
\begin{document}
\title{Calculation Notes}


\maketitle

\section{The problem}

The problem at hand is that we are trying to compute

\begin{equation}
\int dz P(d | z, \text{net infected at } t = 0) P(z | \text{net infected at } t=0)
\end{equation}

Here, $d$ is data, and $z$ are node infection times.  Since we are approaching 2 month on calculing this integral, it is obviously not trivial to compute.  The main reasons are:

\begin{itemize}
\item There are many $z$'s such that $P(z | \text{net infected at } t=0)=0$.  That means doing a simple uniform sampling will be very inefficient.
\item It is unclear how to handle nodes that do not get infected by the end of the observation window, which we have been denoting as $T$.  In other words, how do we sample and compute the probability of certain infection times when some nodes get infected and others do not.
\end{itemize}

To tackle these two problems, we devised 4 algorithms.  Two of the algorithms involve sampling the infection times, $z$ in the interval $[0, \infty]$.  The other two involve sampling $z$ in the interval from $[0, T]$.  The intuition behind the second method is as follows.  Suppose we are working with a net of 4 nodes $v_1, v_2, v_3,$ and $v_4$.  Let $z_i$ be the infection time of node $v_i$.  Furthermore, suppose the $v_1$ is the node that is the initial attacker and therefore $z_1 =0$.  Now, to compute the likelihood, we need to consider the case where all nodes get infected by $T$, only $v_2$ gets infected by $T$, only $v_2$ and $v_3$ gets infected bt $T$... and so on until all possible combinations are exhausted.  To sample over $[0, T]$, we take each of these cases separately.  When doing (uniform) importance sampling, we consider  each \emph{ordering} separately.  For example, the case where  $v_2$ gets infected first, $v_3$ gets infected second and $v_4$ does not get infected will be computed independently from the case where $v_3$ gets infected first, $v_2$ gets infected second and $v_4$ does not get infected.  By proceeding this we, we completely remove any $z$'s that are not allowed given the configuration of the net.  Therefore for any sample of $z$ done this way the value of the integrand will be strictly greater than 0.

\pagebreak

\section{(Uniform) Importance Sampling over $[0, T]$ }

The pseudo code is as follows:

\begin{itemize}
\item $V \leftarrow \{v_i\}$    (The set containing all nodes)
\item  $S \leftarrow$ set of all allowable infection orderings, including the orderings where some nodes did not get infected.  For example an element $s \in S$ might be $\{v_2\}$ or $\{v_3, v_4, v_2\}$. We also write $z_s$ as the infection times that correspond to the order of $s$.  For example, if we sample $s = \{v_2, v_3, v_1\}$ and $z_s = \{3000, 3400, 5000\}$ then the infection time of $v_2=3000$, the infection time of $v_3= 3400$, and the infection time of $v_1 = 5000$.
\item $K \leftarrow 20,000$  (Number of samples of $z_{s_i}$)
\item $lhood \leftarrow 0$
\item \textbf{FOR} $s \in S$           (For each ordering)
\begin{itemize}
\item $plist \leftarrow []$ (Store all samples)
\item $M$ $\leftarrow$ len($s$)  (The number of nodes that are infected)
\item $normconst \leftarrow \frac{T^M}{M!}$
\item \textbf{FOR} $k$ in range($K$)
\begin{itemize}
\item $z_s \leftarrow$ \textbf{sort}(\textbf{rand}($[0,T]^M$))  (Sample infection times)
\item $pz_{infected} \leftarrow P(z_{s}, s | \text{net infected at } t=0)$ (as in ASCII)
\item $pz_{not \; infected} \leftarrow \prod_{j \in V\setminus s} \exp(-\lambda_j (T- \max(z_{s_i})))$
\item $pz \leftarrow pz_{infected}\cdot pz_{not \; infected}$
\item $pdata \leftarrow P(d | z,s, \text{ net infected at } t=0)$ (as in ASCII)
\item $ptotal \leftarrow  pdata \cdot pz$
\item $plist$ \textbf{append}  $ptotal$
\end{itemize}
\item $ lhood \; += \; normcons * \text{ \textbf{mean}}(plist)$
\end{itemize}
\item \textbf{return} $lhood$
\end{itemize}

Note the line $pz_{not \; infected} \leftarrow \prod_{j \in V\setminus s} \exp(-\lambda_j (T- \max(z_{s_i})))$.  This says, for every node $j$ that is not infected by $T$, find the probability of it not getting infected in the time period between the last node that got infected and $T$.  This is just $1-(1-e^{\lambda_j\tau}$ where $\lambda_j$ is the rate at which it receives malicious messages and $\tau$ is just $T-$ time last node gets infected.

\pagebreak

\section{Metropolis Hastings over $[0, \infty]$}

In this case, we do not need to evaluate the integral in parts.  Instead we can propose a $z$ where $\max(z) > T$ and compute $P(d | z, \text{net infected at } t = 0)$ and $P(z | \text{net infected at } t=0)$.  In this MCMC, we are going to accept and reject based on $P(z | \text{net infected at } t=0)$.

Note that similar to the importance sampling case, we are also going to create a set of all possible node orderings so that we do not sample $z$'s that have a probability $0$.  However, unlike the importance sampling, the set of possible orderings only includes the elements where all nodes get infected.  For example, for a net with $N$ nodes, all possible orderings must be of length $N$.

In this algorithm, we will be sampling infection times and infection orderings.  If the infection ordering is $s_i$, then the vector $z_i$ represents the infection times in order of the nodes given by $s_i$.  For example, if we sample $s_1 = \{v_2, v_3, v_1\}$ and $z_1 = \{3000, 3400, 5000\}$ then the infection time of $v_2=3000$, the infection time of $v_3= 3400$, and the infection time of $v_1 = 5000$.

\begin{itemize}
\item $S \leftarrow$ a set whose elements are allowable node infection orderings of length $N$
\item $s_0 \leftarrow $ \textbf{sample}$(S)$ (Sample an ordering)
\item $z_{s_0} \leftarrow$ \textbf{SORT}(\textbf{rand}($[0, T]^N$)) (Sample infection times)
\item $K \leftarrow 100,000$ ( Number of MCMC samples)
\item $\alpha \leftarrow .01$ (Set $\alpha$ value)
\item $\sigma^2 \leftarrow 500$ (Set standard deviation of proposal)
\item $probs \leftarrow []$ (List to store integrand values)
\item for $k$ in \textbf{range}(K)
\begin{itemize}
\item \textbf{if} $\alpha$ $<$ $U(0,1)$
\begin{itemize}
\item $s_1 \leftarrow \textbf{sample}(S)$ (Sample a new infection order)
\end{itemize}
\item \textbf{else}
\begin{itemize}
\item  $s_1 \leftarrow s_0$ (Keep old infection order)
\end{itemize}
\item $z_{s_1} \leftarrow$ \textbf{sort} ($z_{s_0} + \overrightarrow{\mathcal{N}(0, \sigma^2)})$ (Draw proposed infection time)
\item \textbf{If} $\frac{P(z_{s_1}, s_1 | \text{net infected})}{P(z_{s_0}, s_0 | \text{net infected})} > \mathbf{U}(0,1)$ (MH STEP)
\begin{itemize}
\item $s_0 \leftarrow s_1$ (Move to the new infection ordering)
\item $z_{s_0}  \leftarrow z_{s_1}$ (Move to the new infection times)
\end{itemize}
\item \textbf{else}
\begin{itemize}
\item $s_0 \leftarrow s_0$ (Stay at previous infection ordering)
\item $z_{s_0} \leftarrow z_{s_0}$     (Stay at previous infection times)
\end{itemize}
\item $probs$ \textbf{append} $P(d | z_{s_0}, s_0,  \text{net infected at } t=0)$
\end{itemize}
\item $lhood \leftarrow$ \textbf{mean}$(probs)$
\item \textbf{return }$lhood$
\end{itemize}



\end{document}
















\end{document}
