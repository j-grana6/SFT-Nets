\documentclass{article}
\usepackage{amsmath}
\usepackage{color}

\begin{document}

%%%%%%%%%%%%%%%%%%%%%%%%%%%%%
\section{Background-ASCII 1-3}

	Recall that in general, an observation operator is an arbitrary
	noninvertible projection from the space of sequences (the state-space
	transitions and the times they occur). In other words, in the notation
	of our paper, it is a non-invertible function of $(\vec{\sigma},
	\vec{\tau})$. (In math-speak, if we take our observation operator to be
	deterministic, the combination of our observation operator with the
	SFT net is a "unifilar hidden markov process model").  \textcolor{red}{In
	normal-person-speak, we are attempting to find the model that descirbes
	the transmission of messages among nodes.  We know that what we
	observed (the message times) was generated  by applying the observation
	operator to a more complete model.  In other words, the observation operator
	hides from us some variables of the underlying model.  Furthermore, this
	observation operator is not invertible so we can't just apply it to what we observed
	as an attempt to get the underlying model.}

	\textcolor{red}{In general, one way to pick the best model is to use maximum
	likelihood} However our likelihood function gives the likelihood of
	the sequence of fully-specified state-space variables (see Eq. 17
	in our paper). \textcolor{red}{In other words, our likelihood is a
	function of some ``stuff'' that the observation operator hides from us.}

	So to calculate the likelihood of an observation (whatever it is
	conditioned on), we must integrate the expression in Eq. 17 over the
	degrees of freedom that the observation operator projects out. In
	general, a horrible problem.  \textcolor{red}{In our case, this translates
	into integrating over the infection times.} 

	However the scenario that Brian constructed is special ...

	In this scenario, what the observation operator gives is the
	time-stamps at which nodes emit messages. All other information -
	states of nodes at all times, and contents of messages at all times -
	is projected out.

	Moreover, we know what the (stationary) Poisson rate of such emissions
	would be from any node $v$ if $v$ were infected, and what the rate would
	be if $v$ were not infected. So we can write down the likelihood of any
	sequence of emissions generated by $v$ if v were infected, and the
	likelihood if $v$ were not.

	We also make the assumption that once a node gets infected, it
	stays infected.

%%%%%%%%%%%%%%%%%%%%%%%%%%%%%%%%%%
\section{The Likelihood-ASCII 4-8}
	\textcolor{red}{Recall that we are interested in finding the model to describe
	the transmission of messages among nodes.
	Combining the time-stamp assumption of the observation operator and the assumption
	that infected nodes stay infected}, we can write down in closed form the likelihood
        of a sequence of (time-stamps of) emissions from an arbitrary node $v$ across all time, given that 
	$v$ is uninfected up to $z_{v}$, when it gets infected. \textcolor{red}{By following
	the same notation in the paper, we express this likelihood as}
        
        %%% JG: What is \Upsilon here?  Are you trying to write down the likelihood that
        %%% gives P(data | z)?
	\begin{eqnarray*}
		P(\vec{\Upsilon_{v}}, \vec{\Delta\tau_{v}} | \vec{\sigma_{v}(t < z_{v})} = 
			\text{`normal'}, \vec{\sigma_{v}(t \geq z_{v})} = \text{`infected'})
		= \\
		\prod_{n = 1}^{K_{1}} P(\Upsilon_{v}, \Delta \tau (n) | \sigma_{v} = \text{`normal'})
			\cdot  
			\prod_{n = K_{1}}^{K} P(\Upsilon_{v}, \Delta \tau (n) | \sigma_{v} = \text{`infected'})
	\end{eqnarray*}
	
        % Is this ehat you meant?  It is a bit more explicit.  
        \begin{align}
        P(data | z) = \prod_{v\in V}\frac{e^{-\lambda_{v1}z_v}(\lambda_{v1}z_v)^{k_{v1}}}{k_{v1}!}\times 
        \frac{e^{-(\lambda_{v2})(T-z_v)}(\lambda_{v2})(T-z_v)^{k_{v2}}}{k_{v2}!}
        \end{align}
        where $k_{v1}$ is the number of messages emitted from node $v$ before $z_v$, $k_{v2}$ 
        is the number of messages emitted from node $v$ between $z_v$ and $T$, $\lambda_{v1}$ is 
        homogenous poisson transmission rate for node $v$ when it is uninfected and $\lambda_{v2}$ is the 
        homogenous poisson transmission rate for node $v$ when it is infected.  
 
	where $K_{1}$ is the number of emissions from node $v$ before the infected time, $z_{v}$.

	Note that this is conditioning on *part* of what it hidden in the
	output of the observation operator, but not everything.  Formally, to
	write down this likelihood we are actually integrating over the values
	of the other hidden variables concerning the messages from $v$ that we
	cannot observe, namely the specifications of whether those messages
	contain malware or not. But for us that integration is trivial. 
	\textcolor{red}{That said, we are actually only integrating over the infection
	times.}

	However the likelihood that we want to calculate is different 
	\textcolor{red}{because the above assumes the infection times are pre-specified.} 
        For example, in the simplest scenarios we might want
	to test, we would want the likelihood of our entire observation
	sequence either conditioned on the premise that one particular node
	$v'$ is infected at t = 0 \textcolor{red}{corresponding to attacker existence}, or that
	no nodes are ever infected \textcolor{red}{corresponding to no attacker}. Note
	that in this likelihood we do NOT specify the times of infection for
	the other nodes for the case that $v'$ is infected at t = 0. In fact,
	we don't even specify $m$, the number of nodes that get infected. 
	\textcolor{red}{In other words, there might even be some nodes that don't get 
        infected by $T$, something we will discuss below.}

	
	However, we can use equation (1) to calculate the likelihood. To see how, note that
        if we knew the infection times for all nodes $v$ in the net that get infected, 
        $z_{v}$, then we are done because it is then trivial to calculate 
        $P(data|z, attacker)P(z attacker)$ %\textcolor{red}{because in this case the 
	%likelihood can be decomposed as the product of the expression across all infected 
	%nodes in the SFT net.} \textcolor{red}{In math, we say that in a N-node SFT net,
	%where $m \in [1,N]$ nodes get infected in the time interval $[0,T]$, it suffices to
	%specify a vector $\vec{z}$ of infection times such that $\vec{z} \in \cup_{1 \leq 
	% m \leq N}R^{m}$ to fix the likelihood of the data.} 
        However, we do not know $\vec{z}$
	%\textcolor{blue}{dx:because they are random variables?} Yes, that is true.

	Therefore to calculate $P(data  | \text{net infected at } t = 0)$, we need to do an
	integral, integrating out the vector $z$. We can
	decompose this integral as follows:

	\begin{equation}
	   \int dz P(d, z | \text{net infected at } t = 0) = \\
       	\int dz P(d | z, \text{net infected at } t = 0) P(z | \text{net infected at } t=0)
	\end{equation}
	
	\textcolor{red}{Because the first term on the RHS is conditioned on $z$, we can 
	calculate it as we just discussed.} What to do about the second?

	To evaluate the second term in the RHS integrand, note that $z$ fixes
	the sequence of successive node infections. Write that sequence as
	$v(1), v(2), \ldots, v(m)$, so that $v(n)$ specifies what node was the $n^{th}$
	to be infected, where we are assuming that $v(1)$ is infected at $t = 0$ (so
	$z_{v(1)} = 0$), but that no other node is infected then, i.e., for all 
	$i > 1$, $z_{v(i)} > 0$.

	So with our new notation we can write

	\begin{equation}
	 P(z | \text{net infected at } t = 0) = P(\{z_{v(i)}\}, \{v(i)\} | v(1), z_{v(1)} = 0)
	\end{equation}
	where curly brackets indicate a set. 

	Note that with this new notation, we have $\int dz = \sum_{\{v(i)\}} \int dz_{\{v(i)\}}$.
	In particular, that sum includes all subsets of all $N$ nodes that form a directed 
        acyclic path over the network topology, including those subsets that do not involve all $N$
	nodes. But it includes no other subsets.  \red{In other words, we sum over all allowable infection 
        orders.}

	Concentrate on any particular value of $\{v(i)\}$ and $\{z_{\{v(i)\}}\}$ occuring
	inside this sum-interal. For that value we can expand the RHS of Eq. 2
	as a product of N conditional distributions,

	\begin{equation*}
		P(v(2), z_{v(2)} |  v(1), z_{v(1)} = 0)    \times  
		P(v(3), z_{v(3)} |  v(2), z_{v(2)}, v(1), z_{v(1)} = 0)    \times  \ldots
	\end{equation*}

	Next, use the network topology to figure out the set of edges exiting
	$v(1)$ and write it as $C(1)$. Also define $\lambda(i,j)$ as the Poisson rate
	constant for the composite process of \{a non-infection message goes
	from infected node $v(i)$ to a non-infected node $v(j)$, or an infection
	message going from an infected node $v(i)$ to a non-infected node $v(j)$
	and $v(j)$ making a transition to being infected when that message
	arrives\}.  Also define $\lambda(1)$ as $\sum_{j \in C(1)} \lambda(1,j)$,
	i.e., the sum of the rate constants over all edges exiting $v(1)$. Then

	\begin{equation*}
		P(v(2), z_{v(2)} | v(1), z_{v(1)} = 0)  =
		\lambda(1) e^{- \lambda(1) (z_{\{v(2)\}} - z_{\{v(1)\}})} 
		\times \frac{\lambda(1,2)}{\lambda(1)}
	\end{equation*}

	in the usual way (e.g., as in the Gillespie algorithm -- the
	expression on the RHS equals the probability that the first transition
	among all the nodes that are connected to $v(1)$ occurs at the time
	$z_{v(2)}$, times the probability that it is node $v(2)$ that makes that
	transition). So after cancellation, we have
	
	\begin{equation*}
	P(v(2), z_{v(2)} | v(1), z_{v(1)} = 0) =
		\lambda(1, 2) e^{-|C(1)| \lambda \cdot (z_{\{v(2)\}} - z_{\{v(1))\}}}
	\end{equation*}

	where $\lambda$ is the sum of the (homogenous) background traffic rate
	and infection message traffic rate. 

	Note that $\lambda(1, 2) = 0$ if there is no edge going from $v(1)$ to
	$v(2)$. If it were not for this fact, our formula for the conditional
	distribution $P(v(2), z_{v(2)} | v(1), z_{v(1)} = 0)$ would not be
	normalized.

	Similarly, define $C(2)$ as the set of edges exiting either $v(1)$ or
	$v(2)$, and define $\lambda(2) = \sum_{j \in C(2)} \lambda(2,j)$, i.e., the
	sum of the rate constants over all edges exiting $v(2)$. (I'm pretty
	sure that the possibility of a node being connected to both $v(2)$ and
	$v(1)$ doesn't change the fact that this is the correct sum.)  Then use
	our assumption of homogenous (infected node) rate constants to write

	\begin{equation*}
		P(v(3), z_{v(3)} | v(2), z_{v(2)}, v(1), z_{v(1)} = 0)  =  
			K(3) \lambda e^{\lambda(2) (z_{\{v(3)\}} - z_{\{v(2))\}}} 
	\end{equation*}

	where $K(3)$ equals $1$ or $2$, depending on whether under the network
	topology one or both of $v(1)$ and $v(2)$ are connected to $v(3)$, and
	$\lambda$ is the homogenous rate constant.

	We can keep iterating to evaluate the full term on the RHS of Eq. 2, and
	use the result to write down the summand-integrand in Eq. 1. 

	Note that only a tiny fraction of the points in $R^{N}$ are physically
	possible. E.g., we can't have a node $v$ get infected at $t$, and then a
	node $v"$ get infected at $t" > t$, and no other nodes ever get infected,
	if due to the network topology the only way v" can get infected is
	from $v$ via a bottleneck node $v'$ lying between $v$ and $v"$. This will be
	reflected in the likelihood function - all disallowed points in $R^{N}$
	will be have likelihood zero, and furthermore, the likelihood function
	will be properly normalized to account for the contorted shape of the
	set of allowed points.

\section{MCMC Methods-ASCII 9-11 and beyond}

Unfortunately, due to the contorted shape of the subset of
$R^N$ of $z$'s that are actually allowed (given the network topology)
discussed above, I don't think we can do the sum-integral to give our
likelihood in closed form. In fact, even if we fix $\{v(i)\}$, I don't
think we can do the associated integral in closed form. For the same
reason, simple sampling MC with a uniform distribution over $[0, T]^m$
(where m is the number of nodes that get infected in $[0, T]$, specified
by $\{v(i)\}$) may be quite inefficient.

This leaves us with 3 ways to estimate the integral

\subsection{Metropolis Hastings over $[0, \infty]^N$}

The flow is as follows:
\begin{itemize}
\item Create a set $V$ that contains all allowable node infection orderings
\item Initiate an allowable $\{z_0\}$.
\item Initiate $\{v_0(i)\}$
\begin{itemize}
\item for sample in 1:number of mcmc samples
\item if $\alpha$ $<$ $U(0,1)$ : $v_1 = v_0$
\item else : Draw a new ordering, $v_1$ from $V$
\item Define $\{z'_1(i)\} = \{z_0(i) + \mathcal{N}_i(0, \sigma)\}$.
\item Let $\{z_1(i)\}$ be $\{z'_1(i)\}$ in order.
\item Using abusive notation assume that the $j$ th element of $\{z_1(i)\}$ corresponds is the infection time of the $j$ th node to get
infected as given in $\{v(i)\}$.
\item If $\frac{P(z_1 | \text{net infected})}{P(z_0 | \text{net infected})} > \mathbf{U}(0,1)$
\begin{itemize}
\item $z_0 =z_1$
\item $v_0 = v_1$
\end{itemize}
\item Record $z_0$, $v_0$ and $P(D | \{z(i)\}, \text{net infected} )$
\end{itemize}
\item The value of the integral is the average over all of the $P(D | \{z(i)\}, \text{net infected} )$
\end{itemize}

\section{Importance Sampling over $[0, \infty]^N$}

DO a bunch of exponentials...


\end{document}
