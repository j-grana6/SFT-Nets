\documentclass{article}
\usepackage{amsmath}
\usepackage{color}

\begin{document}

%%%%%%%%%%%%%%%%%%%%%%%%%%%%%
\section{Background-ASCII 1-3}

	Recall that in general, an observation operator is an arbitrary
	noninvertible projection from the space of sequences (the state-space
	transitions and the times they occur). In other words, in the notation
	of our paper, it is a non-invertible function of $(\vec{\sigma},
	\vec{\tau})$. (In math-speak, if we take our observation operator to be
	deterministic, the combination of our observation operator with the
	SFT net is a "unifilar hidden markov process model").  \textcolor{red}{In
	normal-person-speak, we are attempting to find the model that descirbes
	the transmission of messages among nodes.  We know that what we
	observed (the message times) was generated  by applying the observation
	operator to a more complete model.  In other words, the observation operator
	hides from us some variables of the underlying model.  Furthermore, this
	observation operator is not invertible so we can't just apply it to what we observed
	as an attempt to get the underlying model.}

	\textcolor{red}{One way to pick the best model is to use maximum
	likelihood} However our likelihood function gives the likelihood of
	the sequence of fully-specified state-space variables (see Eq. 17
	in our paper). \textcolor{red}{In other words, our likelihood is a
	function of some ``stuff'' that the observation operator hides from us.}

	So to calculate the likelihood of an observation (whatever it is
	conditioned on), we must integrate the expression in Eq. 17 over the
	degrees of freedom that the observation operator projects out. In
	general, a horrible problem.  \textcolor{red}{In our case, this translates
	into integrating over the infection times.} \textcolor{blue}{dx: Do we only 
	integrate over the infection times? Two questions: 1. do you mean **observed**
	infection times? Or **actual** infection times (maybe beyond T)? 
	2. Don't we also concern about the contents of the messages?}

	However the scenario that Brian constructed is special ...

	In this scenario, what the observation operator gives is the
	time-stamps at which nodes emit messages. All other information -
	states of nodes at all times, and contents of messages at all times -
	is projected out.

	Moreover, we know what the (stationary) Poisson rate of such emissions
	would be from any node $v$ if $v$ were infected, and what the rate would
	be if $v$ were not infected. So we can write down the likelihood of any
	sequence of emissions generated by $v$ if v were infected, and the
	likelihood if $v$ were not.

	We also make the assumption that once a node gets infected, it
	stays infected.

%%%%%%%%%%%%%%%%%%%%%%%%%%%%%%%%%%
\section{The Likelihood-ASCII 4-8}
	Combining the two assumptions above in our scenario, we might attepmpt to write down
	in closed form the likelihood of a sequence of observed (time-stamps of) emissions
	from an arbitrary node v across all time, given that v is uninfected up to
	time $z_{v}$, when it gets infected. 

	Note that this is conditioning on *part* of what it hidden from the
	observation operator, but not everything.  Formally, to write down
	this likelihood we are actually integrating over the values of the
	other hidden variables concerning the messages from v that we cannot
	observe, namely the specifications of whether those messages contain
	malware or not. But for us that integration is trivial.




\section{MCMC Methods-ASCII 9-11 and beyond}


\end{document}
