\documentclass{article}
\usepackage{amsmath}
\usepackage{color}
\makeatletter
\def\Let@{\def\\{\notag\math@cr}}
\makeatother
\begin{document}

\section{Background-ASCII 1-3}

\subsection{The Observation Operator}

	Recall that in general, an observation operator is an arbitrary
	noninvertible projection from the space of sequences (the state-space
	transitions and the times they occur). In other words, in the notation
	of our paper, it is a non-invertible function of $(\vec{\sigma},
	\vec{\tau})$. (In math-speak, if we take our observation operator to be
	deterministic, the combination of our observation operator with the
	SFT net is a "unifilar hidden markov process model").  In
	normal-person-speak, we are attempting to find the model that descirbes
	the transmission of messages among nodes.  We know that what we
	observed (the message times) was generated  by applying the observation
	operator to a more complete model.  In other words, the observation 
        operator hides from us some variables of the underlying model.  
        Furthermore, this observation operator is not invertible so we can't 
        just apply it to what we observed as an attempt to get the underlying 
        model.

\subsection{Likelihood issues}
        Our likelihood function gives the 
        likelihood of the sequence of fully-specified state-space variables 
        (see Eq. 17 in our paper). This includes message times, the infection times 
        as well as the content of the message.  

	So to calculate the likelihood of an observation (whatever it is
	conditioned on), we must integrate the expression in Eq. 17 over the
	degrees of freedom that the observation operator projects out. In
	general, a horrible problem. In our case, this translates
	into integrating over the infection times. 

\subsection{Our Special Scenario}	
        However the scenario that Brian constructed is special ...

	In this scenario, what the observation operator gives is the
	time-stamps at which nodes emit messages. All other information -
	states of nodes at all times, and contents of messages at all times -
	is projected out.

	Moreover, we know what the (stationary) Poisson rate of such emissions
	would be from any node $v$ if $v$ were infected, and what the rate would
	be if $v$ were not infected. So we can write down the likelihood of any
	sequence of emissions generated by $v$ if $v$ were infected, and the
	likelihood if $v$ were not.

	We also make the assumption that once a node gets infected, it
	stays infected.

%%%%%%%%%%%%%%%%%%%%%%%%%%%%%%%%%%
\section{The Likelihood-ASCII 4-8}
\subsection{If we knew the infection times...}	
    .
	We can write down in closed form the likelihood
        of a sequence of (time-stamps of) emissions from an arbitrary node $v$ 
        across all time, given that  $v$ is uninfected up to $z_{v}$, when it 
        gets infected. 
        
        \begin{align}
        P(data | z) = \prod_{v\in V}\frac{e^{-\lambda_{v1}z_v}(\lambda_{v1}z_v)^{k_{v1}}}{k_{v1}!}\times 
        \frac{e^{-(\lambda_{v2})(T-z_v)}(\lambda_{v2})(T-z_v)^{k_{v2}}}{k_{v2}!}
        \label{nothidden}
        \end{align}
        where $k_{v1}$ is the number of messages emitted from node $v$ before 
        $z_v$, $k_{v2}$ is the number of messages emitted from node $v$ between 
        $z_v$ and $T$, $\lambda_{v1}$ is  homogenous poisson transmission rate for
        node $v$ when it is uninfected and $\lambda_{v2}$ is the  homogenous 
        poisson transmission rate for node $v$ when it is infected.  
 
	Note that this is conditioning on *part* of what it hidden in the
	output of the observation operator, but not everything.  Formally, to
	write down this likelihood we are actually integrating over the values
	of the other hidden variables concerning the messages from $v$ that we
	cannot observe, namely the specifications of whether those messages
	contain malware or not. But for us that integration is trivial. 

\subsection{The problem}
	
        However the likelihood that we want to calculate is different than 
        equation ~\ref{nothidden}.  For example, in the simplest scenarios we might want
	to test, we would want the likelihood of our entire observation
	sequence either conditioned on the premise that one particular node
	$v'$ is infected at t = 0 or that no nodes are ever infected. 
        Note that in this likelihood we do NOT 
        specify the times of infection for the other nodes for the case that 
        $v'$ is infected at t = 0. In fact, we don't even specify $m$, the number
        of nodes that get infected by time $T$.


	However, we can use equation ~\ref{nothidden} as part of the  likelihood. To see how, 
        note that if we knew the infection times for all nodes $v$ in the net 
        that get infected,  $z_{v}$,  we would be done.  In other words, in an 
        N-node SFT net, where we can have between 1 and N nodes get infected in 
        the time interval [0, T], it suffices to specify a point z in the union 
        of spaces $\cup_{1 \le M \le N} R^M$ to fix the likelihood of our data.
        Alas, we do not know that vector z.


	6) Therefore to calculate  $P(data  | \text{net infected at } t = 0)$, 
        we need to do an integral, integrating out the vector $z$. We can 
        decompose this integral as follows:

\begin{align}
\int dz P(d, z | \text{net infected at } t = 0) = \\ 
\int dz P(d | z, \text{net infected at } t = 0) P(z | \text{net infected at } t=0)
\label{main}
\end{align}

	  \textcolor{red}{Again, the first term on the RHS is just equation \ref{nothidden} and 
          is trivial to compute.} What to do about the second?

\subsection{Calculating P(z)}

\emph{SUBSTANTIAL CHANGES 2/21/14}

    One issue that presented itself is that sometimes, all nodes do not get infected within
    the observation window.  To deal with this problem, we have to consider all possible sequences
    of infection times, including those sequences that do not include some of the nodes.  

    More formally, let $D$ be the the set of all sequence of infection times consistent with 
    the cybernet, $M$ is the size of such a sequence, the elements of such a sequence form an 
    $M$-vector $s^M$, and $z^M$ is a vector in $R^M$ all of whose components lie in 
    [0, T]. (Note that we are no longer calling the set $D$ a DAG based on the discussion on
    Feb 20, 2014. It turned out that a graphical representation of node infection orderings
    was not natural).

    What makes the analysis slippery is the that the sizes of the latter
    two random variables $(s^M$ and $z^M)$ are set by the value of the
    first random variable. In other words, if we knew the infection sequence, we would know 
    the size of $s^M$ and $z^M$,  To clarify things, let's transform to a new coordinate system 
    in which both of those latter two random variables are $N$-dimensional:

    To do this, augment the index set of the cybernet nodes, $I = \{1, ..., N\}$, to include a special "null"
    value, $*$. Label that augmented set (which has $N+1$ elements) by $I*$.
    Then for all $M$, map each $s^M \in I^M$ to $s \in (I*)^N$, where each $s_i$ is an element of
    of $I$ if $i <= M$, and all remaining components $s_j$ equal $*$.

    Similarly augment the space $\tau = [0, T]$ to include a special "null" value, $\%$.
    Label that augmented space by $\tau\%$. Then for all $M$, map each $z^M \in \tau^M$ to
    $\bar{z} \in (\tau\%)^N$, where each $\bar{z_i}$ is an element of $\tau$ if $i \leq M$, and all other
    components of $z$ equal $\%$. (Aside: I use the symbol $\bar{z}$, since $z$ already means
    something in the first coordinate system.)

    Note that just as the analysis for the second coordinate system we were not interested in integrating
    all of $\tau^M$, but only the subvolume where $z_1 < z_2 < ...$, so in the this coordinate system
    we will not be interested in integrating over all of $[\tau\%]^N$, but only over the subvolume
    (where until we hit the first component of $\bar{x}$ with a$\%$), $\bar{z_1} < \bar{z_2} < ...$

    In this coordinate system, we can decompose the integral in equation ~\ref{main} as

\begin{equation}
\sum_{M}  \sum_{s}  \int d\bar{z}  [P(d | M, s, \bar{z})  P(\bar{z}, s, M | \text{net infected at t = 0) }]
\end{equation}

where there is an implicit delta function forcing $\bar{z}$ and $s$ to have $(N - M) \%$'s and $*$'s, respectively.

At this point we must do something novel, due to the random variable M. In particular,
we cannot do as we were doing and start by evaluating something like
$P(\bar{z_1}, s_1 | M, \text{net infected at t = 0 })$. The reason is that knowing the total number
of nodes that will get infected before $T$ distorts the probability that $s_1$ is the first
node to get infected, and also distorts the probability that $\bar{z_1}$ is the time it gets infected.
So calculating $P(\bar{z_1}, s_1 | M, \text{net infected at t = 0 })$ is actually quite hard.

One way forward is as follows:

\begin{align}
P(\bar{z}, s, M | \text{ net infected at t = 0 }) = \\ \nonumber
P(M | \bar{z}, s, \text{ net infected at t = 0 })  \times  P(\bar{z}, s | \text{ net infected at t = 0 }) = \\  \nonumber
\delta(M = \text{ the number of non-* components of } s)  \times \\
P(\bar{z}, s | \text{ net infected at t = 0 })
\end{align}


We can do an iterative expansion of $P(\bar{z}, s | \text{net infected at t = 0 })$,
 as

\begin{align}
P(\bar{z_1}, s_1 | \text{ net infected at t = 0 }) \times \\ \nonumber
P(\bar{z_2}, s_2 | \bar{z_1}, s_1, \text{ net infected at t = 0 }) \times ... \\ \nonumber
P(\bar{z_N}, s_N  | ...., \text{ net infected at t = 0 })
\end{align}

where each of those terms is evaluated in terms of a Poisson processes as follows.  


        Use the network topology to figure out the set of edges exiting
	$s_1$ and write it as $C(1)$. Also define $\lambda(i,j)$ as the Poisson rate
	constant for an infectied message going from an infected node $v(i)$ 
        to a non-infected node $v(j)$ and $v(j)$ making a transition to being infected when 
        that message arrives.  Also define $\lambda(1)$ as $\sum_{j \in C(1)} \lambda(1,j)$,
	i.e., the sum of the rate constants over all edges exiting $v(1)$. Then

\begin{equation}
P(s_2, \bar{z_2} | s_1. z_1 =0)  =
\lambda(1) e^{- \lambda(1) (\bar{z_2} - \bar{z_1})} 
\times \frac{\lambda(1,2)}{\lambda(1)}
\end{equation}

        as in the Gillespie algorithm -- the
	expression on the RHS equals the probability that the first transition
	among all the nodes that are connected to $s_1$ occurs at the time
	$\bar{z_2}$, times the probability that it is node $s_2$ that makes that
	transition). 
        
          

        After cancellation we have	
\begin{equation}
P(s_2, \bar{z_2} | s_1, \bar{z_1} = 0) =
\lambda(1, 2) e^{-|C(1)| \lambda \cdot (z_{\{v(2)\}} - z_{\{v(1))\}}}
\label{orig5}
\end{equation}

	where $\lambda$ is the sum of the (homogenous) infection message traffic rate. 

	Note that $\lambda(1, 2) = 0$ if there is no edge going from $s_1$ to
	$s_2$. If it were not for this fact, our formula for the conditional
	distribution $P(s_2, \bar{z_2} | s_1, \bar{z_1} = 0)$ would not be
	normalized.

	Similarly, define $C(2)$ as the set of edges exiting either $s_1$ or
	$s_2$, and define $\lambda(2) = \sum_{j \in C(2)} \lambda(2,j)$, i.e., the
	sum of the rate constants over all edges exiting $s_1$ and $s_2$ (I'm pretty
	sure that the possibility of a node being connected to both $v(2)$ and
	$v(1)$ doesn't change the fact that this is the correct sum.)  Then use
	our assumption of homogenous (infected node) rate constants to write

\begin{equation}
P(s_3, \bar{z_3} | s_2, \bar{z_2} ...0)  =  
K(3) \lambda e^{\lambda(2) (\bar{z_3} - \bar{z_2})} 
\end{equation}

	where $K(3)$ equals $1$ or $2$, depending on whether under the network
	topology one or both of $s_1$ and $s_2$ are connected to $s_3$, and
	$\lambda$ is the homogenous rate constant.

	We can keep iterating to evaluate for all nodes that get infected. Then for nodes that do 
        not get infected we simply use the CDFs whose rate is defined as above but the time
        is given by $T- z_M$.

%%         7) To evaluate the second term in the RHS integrand, note that $z$ fixes
%% 	the sequence of successive node infections. Write that sequence as
%% 	$v(1), v(2), \ldots, v(m)$, so that $v(n)$ specifies what node was the 
%%         $n^{th}$ to be infected, where we are assuming that $v(1)$ is infected 
%%         at $t = 0$ (so $z_{v(1)} = 0$), but that no other node is infected then, 
%%         i.e., for all  $i > 1$, $z_{v(i)} > 0$.

%% 	So with our new notation we can write

%% \begin{equation}
%% P(z | \text{net infected at } t = 0) = P(\{z_{v(i)}\}, \{v(i)\} | v(1), z_{v(1)} = 0)
%% \label{split}
%% \end{equation}
	
%%         where curly brackets indicate a set. 

%% 	Note that with this new notation, the  $\int dz$ in equation ~\eqref{main} 
%%         becomes $ \sum_{\{v(i)\}} \int dz_{\{v(i)\}}$.  i.e. the RHS of ~\eqref{main} 
%%         becomes

%% \begin{align}
%% \sum_{v(i)} \int dz_{v(i)} & P(d | 
%% \{v(i)\}, \{z_{v(i)}\}, \text{net infected at} t  = 0)    \times \\
%% & P(\{v(i)\}, \{z_{v(i)}\} | \text{net infected at} t=0)
%% \end{align}

%%         Note that the sum over $\{v(i)\}$ includes all subsets of the set of $N$
%%         nodes that form a directed acyclic path over the network topology,
%%         including those subsets that do not involve all $N$ nodes. But it
%%         includes no other subsets.

%% 	Concentrate on any particular value of $\{v(i)\}$ and $\{z_{\{v(i)\}}\}$ 
%%         occuring inside this sum-integral. For that value we can expand the RHS of 
%%         equation ~\eqref{split} as a product of N conditional distributions,

%% \begin{align}
%% & P(v(2), z_{v(2)} |  v(1), z_{v(1)} = 0)    \times  \\
%% & P(v(3), z_{v(3)} |  v(2), z_{v(2)}, v(1), z_{v(1)} = 0) \times  \ldots  \times \\
%% & P(\text{no nodes infected in the interval between } z_v(M) \\
%% &\text{ and the end of the window}  |  \{v(i) : 1 \le i \le M\}, z_{v(i) : 1 \le i \le M})
%% \end{align}

	
        %% where T is the time that the observation window ends.

        %% At this point we have all the terms whose product gives us the
        %% conditional distribution we need, and we can use the result to write
        %% down the summand-integrand in Eq. 1.


	Note that only a tiny fraction of the points in $R^{N}$ are physically
	possible. E.g., we can't have a node $v$ get infected at $t$, and then a
	node $v"$ get infected at $t" > t$, and no other nodes ever get infected,
	if due to the network topology the only way v" can get infected is
	from $v$ via a bottleneck node $v'$ lying between $v$ and $v"$. This will be
	reflected in the likelihood function - all disallowed points in $R^{N}$
	will be have likelihood zero, and furthermore, the likelihood function
	will be properly normalized to account for the contorted shape of the
	set of allowed points.

\section{Calculation notes}

        9) Unfortunately, due to the contorted shape of the subset of
        $R^N$ of $\bar{z}$'s that are actually allowed (given the network topology)
        discussed above, I don't think we can do the sum-integral to give our
        likelihood in closed form. In fact, even if we fix $\{s\}$, I don't
        think we can do the associated integral in closed form. For the same
        reason, simple sampling MC with a uniform distribution over $[0, T]^m$
        (where m is the number of nodes that get infected in $[0, T]$, specified
        by $\{s\}$) may be quite inefficient.

        %% As an aside, recall the variable transformation discussed just below
        %% Eq. 5. If we were to make that transformation, then the "contorted
        %% shape of the subset of $R^M$ of $z$'s that are actually allowed (given the
        %% network topology)" is replaced by the subset of $R^M$ in which 
        %% $z_1 \le z_2  ...  \le z_M$ This is a vastly simpler object, one that 
        %% is *independent of network topology*.

        The new coordinate system may allow us to do our calculations much more simply. For
        example, now, for every vector $\{s_i\}$ in the set of allowable infection sequences, the 
        integrand (in the new set of variables) is never zero for any of the
        $\bar{z}$'s,  so even something like importance sampling MC, rather than 
        MCMC, should be possible. 

        \emph{NEW on 2/22/14}

        So under simple sampling we are interested in approximating

\begin{align}
& \sum_{M}  \sum_{s} \int d\bar{z} stuff(M, s, \bar{z}) =\\
& N \sum_M q(M)  V(M) \sum_{s} q(s | M)   (T^M/M!)  \int d\bar{z} q(\bar{z} | M) P(data | \bar{z},s)P(s)P(\bar{z} |s) 
\label{draw}
\end{align}

        where $V(M)$ is the number of infection sequences where  $M$ elemnts are not \%, 
        all three distributions q are uniform over the allowed ranges of their arguments, and 
        "$stuff(M, s, z-bar)$" is the
        (normalized) product of likelihoods discussed above that gives
        $P(d, M, s, \bar{Z} | \text{net infected at t  = 0 })$.

        To do simple sampling we would randomly sample those three distributions,
        proceeding from $q(M)$ to $q(s | M)$ to $q(\bar{z} | M)$. For each $M$ we would
        then sum the associated values of $stuff(M, s, \bar{z})$ for all pairs $(\bar{z}, s)$.
        We would then multiply that sum by $V(M)T^M / M!$
        We would then sum all such products, and multiply by $N$, and be done.

        Note that we can instead generate samples from a uniform distribution over
        infection sequences   -  we just need to use importance sampling (!). This would mean
        introducing the usual correction factors.
 
        The proposal distribution over $M$ that is implicit in uniform sampling over infection sequences is
        $V(M) / |V|$. So if we use that proposal distribution, we must multiply by
        $q(M) / [V(M) / V] = V / NV(M)$ in our integrand. Doing that cancels the $V(M)$
        term on the RHS of Eq. ~\ref{draw}, divides by $N$, and multiplies by $|V|$. Nothing else changes.


\pagebreak

\section{(Uniform) Importance Sampling over $[0, T]$ }

The pseudo code is as follows:

\begin{itemize}
\item $V \leftarrow \{v_i\}$    (The set containing all nodes)
\item  $ S\leftarrow$ set of all allowable infection orderings, including the orderings where some nodes did not get infected.  For example an element $s \in S$ might be $\{v_2, *, *,*\}$
\item $K \leftarrow 20,000$  (Number of samples of per infection sequence)
\item $lhood \leftarrow 0$
\item \textbf{FOR} $s \in S$           (For each ordering)
\begin{itemize}
\item $plist \leftarrow []$ (Store all samples)
\item $M$ $\leftarrow$ len($s$)  (The number of nodes that are infected)
\item $normconst \leftarrow \frac{T^M}{M!}$
\item \textbf{FOR} $k$ in range($K$)
\begin{itemize}
\item $\bar{z} \leftarrow$ \textbf{sort}(\textbf{rand}($[0,T]^M$))  (Sample infection times)
\item $pz_{infected} \leftarrow P(\bar{z}, s | \text{net infected at } t=0)$ 
\item $pdata \leftarrow P(d | z,s, \text{ net infected at } t=0)$ (as in ASCII)
\item $ptotal \leftarrow  pdata \cdot pz$
\item $plist$ \textbf{append}  $ptotal$
\end{itemize}
\item $ lhood \; += \; normcons * \text{ \textbf{mean}}(plist)$
\end{itemize}
\item \textbf{return} $lhood$
\end{itemize}

Note that we are not explicitely sampling the $q$ in equation ~\ref{draw}. However, this can be easily changed to 
use a non-uniform importance sampling approach.

\pagebreak

\subsection{MCMC over $[0, \infty]$}

\begin{itemize}
\item $s_0 \leftarrow $ \textbf{sample}$(S)$ (Sample an ordering)
\item $\bar{z}_0 \leftarrow$ \textbf{SORT}(\textbf{rand}($[0, T]^N$)) (Sample infection times)
\item $K \leftarrow 500,000$ ( Number of MCMC samples)
\item $\sigma^2 \leftarrow 500$ (Set standard deviation of proposal)
\item $probs \leftarrow []$ (List to store integrand values)
\item for $k$ in $\textbf{range}(K)$
\begin{itemize}
\item $\bar{z}_1 \leftarrow  \bar{z}_0 + \overrightarrow{\mathcal{N}(0, \sigma^2)}$ (Draw proposed infection time)
\item (The new ordering is implied by the draw of $\bar{z}_1$
\item \textbf{If} $\frac{P(\bar{z}_1, s_1 | \text{net infected})}{P(\bar{z}_0, s_0 | \text{net infected})} > \mathbf{U}(0,1)$ (MH STEP)
\begin{itemize}
\item $s_0 \leftarrow s_1$ (Move to the new infection ordering)
\item $\bar{z}_0  \leftarrow \bar{z}_1$ (Move to the new infection times)
\end{itemize}
\item \textbf{else}
\begin{itemize}
\item $s_0 \leftarrow s_0$ (Stay at previous infection ordering)
\item $\bar{z}_0 \leftarrow \bar{z}_0$     (Stay at previous infection times)
\end{itemize}
\item $probs$ \textbf{append} $P(d | \bar{z}_0, s_0,  \text{net infected at } t=0)$
\end{itemize}
\item $lhood \leftarrow \textbf{mean}(probs)$
\item \textbf{return }$lhood$
\end{itemize}

This does not take into account sampling over $s$ since as we discussed before, the last implementation did not satisfty detailed balance.

%% \subsection{Metropolis Hastings over $[0, \infty]^N$}

%% The flow is as follows:
%% \begin{itemize}

%% \item Initiate $\bar{z}$
%% \item Initiate a set $s$
%% \begin{itemize}
%% \item for sample in 1:number of mcmc samples
%% \item Sample $\bar{z}\} = \{\bar{z}_{v(i)_0} + \mathcal{N}_i(0, \sigma)\}$
%% \item Define $z_1 = \{z_{v(i)_1}\} \cap \{v_1(i)\}$ 
%% \item If $\frac{P(z_1 | \text{net infected})}{P(z_0 | \text{net infected})} > \mathbf{U}(0,1)$
%% \begin{itemize}
%% \item $z_0 =z_1$
%% \item $v_0 = v_1$
%% \end{itemize}
%% \item else: $z_0, v_0$ don't change.
%% \item Record $z_0$, $v_0$ and $P(D | \{z_0(i)\}, \text{net infected} )$
%% \end{itemize}
%% \item The value of the integral is the average over all of the $P(D | \{z(i)\}, \text{net infected} )$
%% \end{itemize}

%% In fact, we do not have to draw node orderings and infection times separately.  Instead, we can just draw $z$ and if the node ordering is not permitted by the topology, we assign it probability 0 and re-sample.  

%% \subsection{Importance Sampling over $[0, \infty]^N$}

%% The importance sampling over $[0, \infty]^N$ will start similarly to the Metropolis-Hastings by sampling infection order.

%% A quick review on importance sampling.  Suppose we want to approximate
%% \begin{align}
%% \int dx f(x)p(x)
%% \end{align}

%% If we can sample from $p(x)$ we can just sample $X_1. X_2 ...X_N$ from $p$ and compute $\frac{1}{N}\sum_{i=1}^Nf(X_i)$.
%% However, if we cannot sample from $p(x)$ but can sample from some other distribution $q(x)$, we can sample $X'_1, X'_2 ...X'_N$ from 
%% $q(x)$ and compute $\frac{1}{N}\sum_{i=1}^Nw(X'_i)$ where $w(X'_i) = \frac{f(X'_i)p(X'_i)}{q(X'_i)}$.  

%% In our case, the importance sampling algorithm will be as follows:


%% \begin{itemize}
%% \item Create a set $V$ that contains all allowable node infection orderings
%% \item Initiate an infection order
%% \item for $\mathbf{samp}$ in 1: imp samps
%% \begin{itemize}
%% \item if $\alpha$ $<$ $U(0,1)$ : keep the same infection order
%% \item else : Draw a new infection order
%% \item Draw $z_{v(j)}$ (the infection time of the $j$th node to be infected) recursively
%% as $z_{v(j)} = \sum_{i=1}^{j-1}z_{v(i)} + \mathbf{EXP}(\sum_{i=1}^{j-1}\lambda^{'}_{i,j})$ 
%% where $\lambda^{'}_{i,j}$ is the (possibly 0) rate of malicious message transmission from $i$ to $j$,  and $\mathbf{EXP}(y)$ is 
%% an exponential random variable with rate parameter $y$
%% \item Given a draw of $z$, we can compute $q(z)$ as $\prod_{v(j)\in V}(\sum_{i=1}^{j-1}\lambda^{'}_{i,j})\exp^{-((\sum_{i=1}^{j-1}\lambda^{'}_{i,j})\Delta z_v(j)}$
%% where $\Delta z_{v(j)} = z_{v(j)} - z_{v(j-1)}$.
%% \item Record $X_{\mathbf{samp}} = \frac{P(data |z, \text{infected})P(z|\text{infected})}{q(z)}$
%% \end{itemize}
%% \item The integral approximation is then $\frac{1}{\text{imp samps}}\sum_{i=1}^{imp samps}X_i$
%% \end{itemize}

%% \subsection{Metropolis Hastings Over  $[0, T]$}
%%         This method is a bit trickier and involves summing over DAGs.  
%%         As an example, consider the sum-integral over all possible $\{v(i)\}$ and
%%         associated $R^M$ vectors $z_{v(i)}$ giving our likelihood:
%% \begin{equation}
%% \sum_{v(i)} \int dz_{v(i)} P(d | \{v(i)\}, z_{v(i)}, \text{ net infected at t } = 0)  P(\{v(i)\}, z_{v(i)} | \text{ net infected at } t=0)
%% \label{sumdags}
%% \end{equation}

%%         (Note that depending on how many nodes get infected, as specified by
%%         $\{v(i)\}$, the dimension of $z_{v(i)}$ ranges from 0 to N - 1.).  
%%         Also remember that to compute the 2nd term on the right hand side of
%%         equation ~\eqref{sumdags}, we need to compute the probability of nodes
%%         $i >M$ not getting infected by time $T$.

%%         To approximate this sum-integral, say we choose the target
%%         distribution of the MCMC to be 
%%         $P(\{v(i)\}, z_{v(i)} | \text{net infected } t=0).$
%%         So we are interested in averaging 
%%         $P(d | \{v(i)\}, z_{v(i)}, \text{net infectedat } t = 0)$
%%         over all pairs $(\{v(i)\}, z_{v(i)})$ generated (i.e., kept) in
%%         the MCMC that has as its target distribution the conditional
%%         distribution $P(\{v(i)\}, z_{v(i)} | \text{net infected } t=0)$.

%%         Then we could have the proposal distribution be the following:

%%         Given a current vector $\{v(i)\}, \{z_v(i)\}$, with some fixed 
%%         probability alpha, do the following:

%%         i) leave {v(i)} unchanged, and then generate a new set of values
%%         $\{z'_v(i)\}$ by the following procedure:
%%         Uniformly randomly sample $[0, T]$ a total of $M$ times. Order
%%         those $M$ values from the lowest to the highest. These are the new
%%         values of the elements in the set $\{z'_v(i)\}$, ordered in order of
%%         increasing $i$.

%%         ii) Form a new directed acyclic path through the network topology by
%%         starting at $v(1)$, randomly choosing one of its outgoing edges to get
%%         $v(2)$, with fixed probability gamma stopping, and if we don't stop,
%%         randomly choose from  the set of outgoing edges from $v(2)$ that do
%%         not lead to $v(1)$ (unless that set is empty, in which case we stop).

%%         Note that this sampling distribution is not symmetric over all pairs
%%         $(\{v(i)\}, z_{v(i)})$, so we must use full Metropolis-Hastings.

%%         To be precise, after forming a sample of this proposal distribution,
%%         we would keep / reject the associated pair $((\{v(i)\}, z_{v(i)})$
%%         according to the associated value of 
%%         $P(\{v(i)\}, z_{v(i)} | \text{net infected } t=0)$ 
%%         (a value given by the calculation  in point (7) above) combined with an 
%%         explicit calculation of what the probability of generating that new sample 
%%         pair is (under the rule for generating pairs given just above). We would 
%%         then average the associated  value of the quantity 
%%         $P(d | \{v(i)\}, z_{v(i)}, \text{net infected at } t = 0)$ over all kept
%%         pairs.

%%         ===== Note I *think* that we can evaluate the probability of
%%         generating a new sample pair from a current one up to an overall
%%         normalization constant, as required by MH, but haven't fully checked.

%%         The pseudo code for this above method is as follows, but with one small
%%         change.  Instead of sampling $\{v(i)\}$ as above, we can just directly sample
%%         the set of all possible $\{v(i)\}$.  I think this would make the proposal 
%%         distribution symmetric.  

%% \begin{itemize}
%% \item Create a set $V$ that contains all allowable node infection orderings of 
%% size $M \le N$.
%% \item Initiate $\{v_0(i)\}$
%% \item Initiate an allowable $\{z_{v(i)_0}\}$.
%% \item Define $z_0 = \{z_{v(i)_0}\} \land \{v_0(i)\}$ 
%% \begin{itemize}
%% \item for sample in 1:number of mcmc samples
%% \item if $\alpha$ $<$ $U(0,1)$ : $\{v_1(i)\} = \{v_0(i)\}$
%% \item else : Draw a new ordering, $\{v_1(i)\}$ from $V$
%% \item Sample $\{z_{v(i)_1}\} = \{z_{v(i)_0} + \mathcal{N}_i(0, \sigma)\}$ for $i=2 ..M'$
%% \item Define $z_1 = \{z_{v(i)_1}\} \land \{v_1(i)\}$ 
%% \item If $\frac{P(z_1 | \text{net infected})}{P(z_0 | \text{net infected})} > \mathbf{U}(0,1)$
%% \begin{itemize}
%% \item $z_0 =z_1$
%% \item $v_0 = v_1$
%% \end{itemize}
%% \item else: $z_0, v_0$ don't change.
%% \item Record $z_0$, $v_0$ and $P(D | \{z_0(i)\}, \text{net infected} )$
%% \end{itemize}
%% \item The value of the integral is the average over all of the $P(D | \{z(i)\}, \text{net infected} )$
%% \end{itemize}

%% The only difference between this and the Metropolis Hastings over $[0, \infty]$ is that in this case, we are only sampling infection times that are less than $T$ and explicitly computing the probability of the other nodes not getting infected.  

%% \subsection{Importance Sampling over $[0, T]$ }
%% This method uses the same idea as the Metropols Hastings over $[0, T]$ (i.e. the summing DAGS method) but instead uses importance sampling.  We also take advantage of being able to explicitly compute the probability of witnessing uninfected nodes.   

%% The pseudo code is as follows:

%% \begin{itemize}
%% \item Create a set $V$ that contains all allowable node infection orderings of 
%% length $M \le N$
%% \item Initiate an infection order
%% \item for $\mathbf{samp}$ in 1: imp samps
%% \begin{itemize}
%% \item if $\alpha$ $<$ $U(0,1)$ : keep the same infection order of length $M$
%% \item else : Draw a new infection order $M$
%% \item Draw $M$ times from $[0,T]$.  Order them.  These are the infection times
%% \item Given a draw of $z$, we can compute $q(z)$ as $\frac{1}{T}^M \times$ Probability of selecting the specific $\{v(i)\}$ 
%% \end{itemize}
%% \item For each $\{v_i\}$ compute the average of the sampled 
%% $P(data | z_v{i}, \text{ attacker})P(z | attacker)/q(z)$.
%% \item The value of the integral is the sum of over $\{v(i)\}$ in the previous step.
%% \end{itemize}
%% The issue here is the exact computation of $q(z)$.  How do we compute the probability
%% of sampling a specific $v\{i\}$ and include that in the importance sampling.  

%% \end{document}

%% should we try uniform sampling?

%% \section{Closed Form}

%% Here we assume the $\bar{z_1} = 0$.  That is, some node is infected at t=0

%% \begin{align}
%% & \int_zdzP(d| z)*P(z) \\
%% & = \int d\bar{z}dsP(d|\bar{z}, s)P(s, \bar{z}) \\
%% & = \int d\bar{z}dsP(d|\bar{z}, s)P(s)P(\bar{z} | s) \\
%% & \text { Since the region of S is discrete ...}\\
%% & =\sum_s\int dz P(d|\bar{z}, s)P(s)P(\bar{z} | s) \\
%% & = \sum_sP(s) \int dz P(d|\bar{z}, s)P(\bar{z} | s) \\
%% & \text{For a given } s \text{ let } \gamma_{i,s} \text{ be the rate of infected messages into node } s(i) \\
%% & \text{ given that  nodes } s(i'), i'<i \text{ are also infected.  Then we have } \\
%% & = \sum_sP(s) \int dz P(d|\bar{z}, s)\prod_{i=2}^N\gamma_{i,s}e^{-\gamma_{i,s}(z_i - z_{i-1})} \\
%% & = \sum_sP(s)\prod_{i=2}^N\gamma_{i,s} \int dz P(d|\bar{z}, s)\exp(\sum_{i=2}^{N-1}z_i(\gamma_{i+1,s} - \gamma_{i, s}) + \gamma_{N,s} z_N) \\
%% & \text{ Let } \lambda_{1,i}, \lambda_{2,1}, k_i, N_i \text { Be the rate of clean message transmission, the sum of clean and malicious message transmission, the number of messages emitted before 

%% HERE IS WHY WE CAN'T HAVE CLOSED FORM.  THE K PARAMETER IN THE POISSON PROCESS DEPENDS ON Z>


%% & =  \sum_sP(s)\prod_{i=2}^N\gamma_{i,s} \times \\
%% & \int dz \prod_i\big[(\frac{\lambda_{1,i}z_i)^{k_i}\exp(-\lambda_{1,i}z_i)}{k_i!}(\frac{\lambda_{1,2}(T-z_i)^{N_i - k_i}\exp(-\lambda_{1,2}(T-z_i)}{N_i-k_i!}\big] \exp(\sum_{i=2}^{N-1}z_i(\gamma_{i+1,s} - \gamma_{i, s}) + \gamma_{N,s} z_N) \\
%% \end{align}


\end{document}
